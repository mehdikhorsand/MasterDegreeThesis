% !TeX root=../main.tex
% در این فایل، عنوان پایان‌نامه، مشخصات خود، متن تقدیمی‌، ستایش، سپاس‌گزاری و چکیده پایان‌نامه را به فارسی، وارد کنید.
% توجه داشته باشید که جدول حاوی مشخصات پروژه/پایان‌نامه/رساله و همچنین، مشخصات داخل آن، به طور خودکار، درج می‌شود.
%%%%%%%%%%%%%%%%%%%%%%%%%%%%%%%%%%%%
% دانشگاه خود را وارد کنید
\university{دانشگاه تهران}
% پردیس دانشگاهی خود را اگر نیاز است وارد کنید (مثال: فنی، علوم پایه، علوم انسانی و ...)
\college{پردیس دانشکده‌های فنی}
% دانشکده، آموزشکده و یا پژوهشکده  خود را وارد کنید
\faculty{دانشکدهٔ برق و کامپیوتر}
% گروه آموزشی خود را وارد کنید (در صورت نیاز)
\department{گروه نرم‌افزار}
% رشته تحصیلی خود را وارد کنید
\subject{مهندسی کامپیوتر}
% گرایش خود را وارد کنید
\field{مهندسی نرم‌افزار}
% عنوان پایان‌نامه را وارد کنید
\title{تولید خودکار آزمایه برای سامانه‌های پردازش تراکنش‌های مالی بر مبنای روش آزمون تصادفی تطبیقی}
% نام استاد(ان) راهنما را وارد کنید
\firstsupervisor{دکتر رامتین خسروی}
\firstsupervisorrank{استادیار}
%\secondsupervisor{دکتر راهنمای دوم}
%\secondsupervisorrank{استادیار}
% نام استاد(دان) مشاور را وارد کنید. چنانچه استاد مشاور ندارید، دستورات پایین را غیرفعال کنید.
%\firstadvisor{دکتر مشاور اول}
%\firstadvisorrank{استادیار}
%\secondadvisor{دکتر مشاور دوم}
% نام داوران داخلی و خارجی خود را وارد نمایید.
\internaljudge{دکتر سیامک محمدی}
\internaljudgerank{دانشیار}
\externaljudge{دکتر مجتبی وحیدی اصل}
\externaljudgerank{استادیار}
\externaljudgeuniversity{دانشگاه شهید بهشتی}
% نام نماینده کمیته تحصیلات تکمیلی در دانشکده \ گروه
\graduatedeputy{دکتر سیامک محمدی}
\graduatedeputyrank{دانشیار}
% نام دانشجو را وارد کنید
\name{مهدی}
% نام خانوادگی دانشجو را وارد کنید
\surname{خرسند}
% شماره دانشجویی دانشجو را وارد کنید
\studentID{810100339}
% تاریخ پایان‌نامه را وارد کنید
\thesisdate{شهریور ۱۴۰۳}
% به صورت پیش‌فرض برای پایان‌نامه‌های کارشناسی تا دکترا به ترتیب از عبارات «پروژه»، «پایان‌نامه» و «رساله» استفاده می‌شود؛ اگر  نمی‌پسندید هر عنوانی را که مایلید در دستور زیر قرار داده و آنرا از حالت توضیح خارج کنید.
%\projectLabel{پایان‌نامه}

% به صورت پیش‌فرض برای عناوین مقاطع تحصیلی کارشناسی تا دکترا به ترتیب از عبارت «کارشناسی»، «کارشناسی ارشد» و «دکتری» استفاده می‌شود؛ اگر نمی‌پسندید هر عنوانی را که مایلید در دستور زیر قرار داده و آنرا از حالت توضیح خارج کنید.
%\degree{}
%%%%%%%%%%%%%%%%%%%%%%%%%%%%%%%%%%%%%%%%%%%%%%%%%%%%
%% پایان‌نامه خود را تقدیم کنید! %%
\dedication
{
{\Large تقدیم به:}\\
\begin{flushleft}{
	\huge
خانواده عزیزم
}
\end{flushleft}
}
%% متن قدردانی %%
%% ترجیحا با توجه به ذوق و سلیقه خود متن قدردانی را تغییر دهید.
\acknowledgement{

سپاس خداوندگار حکیم را که با لطف بی‌کران خود، آدمی را به زیور عقل آراست.

در آغاز وظیفه‌  خود  می‌دانم از زحمات بی‌دریغ استاد راهنمای خود،  جناب آقای دکتر رامتین خسروی، صمیمانه تشکر و  قدردانی کنم که در طول انجام این پایان‌نامه با نهایت صبوری همواره راهنما و مشوق من بودند و قطعاً بدون راهنمایی‌های ارزنده‌ ایشان، این مجموعه به انجام نمی‌رسید.


%از همکاری و مساعدت‌های دکتر ... مسئول تحصیلات تکمیلی و سایر کارکنان دانشکده بویژه سرکار خانم ... کمال تشکر را دارم.

%با سپاس بی‌دریغ خدمت دوستان گران‌مایه‌ام، خانم‌ها ... و آقایان ... در آزمایشگاه ...، که با همفکری مرا صمیمانه و مشفقانه یاری داده‌اند.

و در پایان، بوسه می‌زنم بر دستان خداوندگاران مهر و مهربانی، پدر و مادر عزیزم و بعد از خدا، ستایش می‌کنم وجود مقدس‌شان را و تشکر می‌کنم از خانواده عزیزم به پاس عاطفه سرشار و گرمای امیدبخش وجودشان، که بهترین پشتیبان من بودند.
}
%%%%%%%%%%%%%%%%%%%%%%%%%%%%%%%%%%%%
%چکیده پایان‌نامه را وارد کنید
\fa-abstract{

روش آزمون تصادفی تطبیقی برای رفع کاستی‌های روش آزمون تصادفی، مانند تکرارپذیری، ناکارآمدی و تمرکز محدود، در سال ۲۰۰۱ توسط چِن و همکارانش ابداع شد. در ابتدا، این روش روی نرم‌افزارهایی با دامنه ورودی عددی متمرکز شد و با استفاده از انتخاب آزمایه‌های پراکنده روی دامنه ورودی، سعی در رفع اشکالات روش آزمون تصادفی داشت. 
در این روش، برای انتخاب هر آزمایه جدید، تعدادی آزمایه کاندید به صورت تصادفی تولید می‌شد و آزمایه‌ای که بیشترین تفاوت را با مجموعه آزمایه‌های انتخاب‌شده قبلی داشت، انتخاب می‌گردید. میزان تفاوت بین آزمایه‌ها ابتدا در دامنه‌های عددی با نگاشت ورودی‌های نرم‌افزار روی یک فضای چندبعدی و سپس محاسبه فاصله بین نقاط متناظر آزمایه‌ها محاسبه می‌شد.
به دلیل نتایج بهتری که این روش نسبت به روش آزمون تصادفی در دامنه‌های عددی از خود نشان داد، پژوهش‌های متعددی در زمینه پیاده‌سازی روش آزمون تصادفی تطبیقی روی دامنه‌های غیرعددی با ورودی‌های شئ‌گرا انجام شد. نقطه اشتراک تمامی این روش‌ها، ارائه یک راهکار برای نگاشت آزمایه‌های شئ‌گرا روی یک یا چند فضای چندبعدی با استفاده از ورودی‌ها و خروجی‌های سیستم تحت آزمون در حین اجرای آزمایه‌ها و همچنین ارائه یک راهکار برای کاهش هزینه انتخاب آزمایه در تعداد آزمایه‌های بالا بوده‌است. از آنجایی که میزان تفاوت هر آزمایه کاندید باید با هر آزمایه درون مجموعه آزمایه‌های انتخاب‌شده قبلی محاسبه شود، هنگامی که تعداد آزمایه‌های درون مجموعه آزمایه‌های انتخاب‌شده قبلی زیاد است، انتخاب آزمایه جدید هزینه زیادی خواهد داشت.
در این پژوهش، روشی برای محاسبه میزان تفاوت بین آزمایه‌ها مبتنی بر استراتژی جدید امتیازدهی به رفتارهای مشاهده‌شده توسط سیستم تحت آزمون در هنگام اجرای آزمایه ارائه شده است. این روش علاوه بر توانایی محاسبه میزان تفاوت بین دو آزمایه، امکان محاسبه میزان تفاوت بین یک آزمایه و یک مجموعه آزمایه دیگر و یا حتی محاسبه میزان تفاوت بین دو مجموعه آزمایه مجزا را دارد و هزینه انتخاب آزمایه جدید در این روش به تعداد آزمایه‌های انتخاب‌شده قبلی وابسته نیست و عملکرد بهتری با توجه به معیارهای ارزیابی میزان پوشش و معیارهای ارزیابی میزان قدرت تشخیص اشکال نسبت به سایر روش‌های انتخاب آزمایه در روش آزمون تصادفی تطبیقی از خود نشان داده است.
}
% کلمات کلیدی پایان‌نامه را وارد کنید
%\keywords{حداکثر ۵ کلمه یا عبارت، متناسب با عنوان، قالب پایان‌نامه، لاتک}
\keywords{
	آزمون نرم‌افزار،
	تولید آزمایه به صورت خودکار، 
	‌ آزمون تصادفی، 
	 آزمون تصادفی تطبیقی
	}
% انتهای وارد کردن فیلد‌ها
%%%%%%%%%%%%%%%%%%%%%%%%%%%%%%%%%%%%%%%%%%%%%%%%%%%%%%
