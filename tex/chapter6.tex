% !TeX root=../main.tex
\chapter{بحث و نتیجه‌گیری}\label{chapter6}

در این پژوهش، الگوریتم‌های تصادفی تطبیقی از لحظه ابداع تا آخرین پژوهش انجام‌شده به‌صورت خلاصه و کامل در فصل دو توضیح داده شده و متوجه شدیم که هدف این الگوریتم‌ها انتخاب آزمایه‌های پراکنده روی دامنه ورودی سیستم تحت آزمون است تا اثربخشی و کارایی مجموعه آزمایه‌های نهایی افزایش یابد.
مجموعه پژوهش‌هایی که تاکنون در این زمینه انجام شده‌اند، به‌صورت خلاصه و دسته‌بندی‌شده شرح داده شده‌اند و نحوه کار آخرین ایده پیاده‌سازی آزمون تصادفی تطبیقی که نسبت به روش‌های قبلی عملکرد بهتری از خود نشان داده است، به‌صورت کامل و دقیق توضیح داده شد.

سپس در فصل \ref{chapter4}، روال کار روش پیشنهادی انتخاب آزمایه مبتنی بر امتیازدهی به‌صورت دقیق و مرحله‌به‌مرحله با ذکر چند مثال توضیح داده شد و شباهت‌ها و تفاوت‌های آن با روال کار روش‌های انتخاب آزمایه مبتنی بر محاسبه فاصله بررسی گردید.
همچنین مشاهده شد که روش ارائه‌شده در این پژوهش انعطاف‌پذیری بالایی دارد و آزمونگر امکان تعریف معیارهای ارزیابی آزمایه متفاوتی برای انتخاب آزمایه در این روش را دارد.

در نهایت، در فصل \ref{chapter5}، عملکرد روش پیشنهادی در این پژوهش با آخرین ایده پیاده‌سازی روش آزمون تصادفی تطبیقی مقایسه شد و مشاهده شد که روش پیشنهادی در این پژوهش نسبت به آخرین ایده پیاده‌سازی روش آزمون تصادفی تطبیقی، در معیارهای ارزیابی میزان پوشش و قدرت تشخیص اشکال عملکرد بهتری از خود نشان می‌دهد. همچنین مشخص شد که هزینه زمانی و فضایی روش پیشنهادی در این پژوهش، برخلاف سایر روش‌های پیاده‌سازی آزمون تصادفی تطبیقی، به تعداد آزمایه‌های انتخاب‌شده قبلی وابسته نیست.
\section{جمع‌بندی نتایج}

برای جمع‌بندی، روش‌های انتخاب آزمایه مبتنی بر محاسبه فاصله نسبت به روش‌های انتخاب آزمایه مبتنی بر امتیازدهی، برتری‌های زیر را دارند.

\begin{itemize}
	
	\item[\checkmark] عملکرد بهتر در پوشش کد سیستم تحت آزمون.
	\item[\checkmark] قدرت تشخیص اشکال بیشتر و مشاهده شکست با تعداد آزمایه‌های کمتر و در زمان کمتر.
	\item[\checkmark] اولین روش آزمون تصادفی تطبیقی که هزینه فضایی و زمانی آن به تعداد مجموعه آزمایه‌های تولیدشده قبلی وابسته نیست.
	\item[\checkmark] انعطاف‌پذیری بالا که امکان تعریف معیارهای امتیازدهی جدید را به آزمونگر می‌دهد.
	\item[\checkmark] جامعیت استراتژی پیشنهادی و امکان پیاده‌سازی روی انواع مختلف سیستم‌های نرم‌افزاری.

\end{itemize}

\section{گسترش ایده انتخاب آزمایه مبتنی بر امتیازدهی}

تاکنون برتری‌های روش آزمون تصادفی تطبیقی مبتنی بر امتیازدهی نسبت به روش آزمون تصادفی تطبیقی مبتنی بر محاسبه فاصله از جهات و دیدگاه‌های مختلف بررسی شده‌اند و روش انتخاب آزمایه مبتنی بر امتیازدهی به‌عنوان یک روش جدید و کارآمد برای پیاده‌سازی آزمون تصادفی تطبیقی معرفی شده است. در ادامه، چند نمونه از کارهایی که می‌توان برای بهبود و گسترش بیشتر این روش انجام داد، بررسی خواهیم کرد.

\subsection{کاهش هزینه انتخاب آزمایه}

همان‌طور که مشاهده کردید، برای انتخاب آزمایه باید لیست رفتارهای مشاهده‌شده توسط هر آزمایه را در اختیار داشته باشیم و آن‌ها را با لیست رفتارهای مشاهده‌شده توسط آزمایه‌های انتخاب‌شده قبلی مقایسه کنیم. به ازای هر رفتار درون لیست رفتارهای مشاهده‌شده توسط آزمایه کاندید، باید تعداد دفعاتی که آن رفتار توسط آزمایه‌های قبلی مشاهده شده است را بدانیم تا بتوانیم امتیاز و ارزش آن رفتار را تعیین کنیم.

در این پژوهش، فرآیند مقایسه و پیدا کردن یک رفتار درون لیست رفتارها به‌صورت خطی انجام شده است و برای هر بار یافتن یک رفتار درون لیست رفتارها، تمام اعضای لیست مشاهده می‌شوند.

با توجه به تعریف رفتار در عناصر تکرارشونده، رفتار هر عنصر تکرارشونده ترتیبی از عملیات‌های تعریف‌شده است که آن عنصر تکرارشونده در فرآیند اجرای خود انجام می‌دهد. دلیل مشاهده رفتارهای متفاوت برای یک عنصر تکرارشونده، وجود شرط‌ها درون کد آن عنصر است. در واقع، اگر درون کد یک عنصر تکرارشونده شرطی وجود نداشته باشد، آن عنصر تنها یک رفتار خواهد داشت و رفتار آن همیشه ثابت خواهد بود.

با توجه به این نکته، می‌توان رفتارهای مشاهده‌شده توسط آزمایه‌ها را به جای لیست، در قالب یک گراف یا درخت ذخیره کرد. در این صورت، دیگر نیازی به مشاهده تمام رفتارهای درون یک لیست برای یافتن یک رفتار نیست که این امر می‌تواند در نهایت موجب کاهش هزینه فضایی و زمانی الگوریتم انتخاب آزمایه مبتنی بر امتیازدهی شود.

\subsection{ارائه ابزار}

در روش آزمون تصادفی تطبیقی مبتنی بر امتیازدهی، با توجه به فلوچارت نمایش داده شده در شکل \ref{scoringflowcahrt}، در ابتدا چند آزمایه تصادفی به‌صورت خودکار تولید می‌شود و سپس این آزمایه‌ها روی سیستم تحت آزمون اجرا می‌شوند و رفتارهای مشاهده‌شده توسط هر آزمایه به‌دست می‌آید. سپس با توجه به رفتارهای مشاهده‌شده توسط هر آزمایه کاندید و مجموعه رفتارهای مشاهده‌شده توسط آزمایه‌های قبلی، امتیاز هر آزمایه کاندید محاسبه شده و آزمایه با بیشترین امتیاز انتخاب می‌شود.

تنها بخشی که نیاز به دخالت آزمونگر دارد، اضافه کردن دستوراتی درون کد سیستم تحت آزمون است که کلاس ضبط‌کننده رفتارها را فراخوانی می‌کند. برای این که این کار بدون دخالت مستقیم آزمونگر و به‌صورت خودکار انجام شود، می‌توان راهکارهایی اندیشید.

یکی از این راهکارها استفاده از کتابخانه‌هایی است که می‌توان مراحل اجرای کد را با آن‌ها ردیابی کرد؛ مانند کتابخانه تِرِیس
\LTRfootnote{trace - \hyperlink{https://docs.python.org/3/library/trace.html}{docs.python.org/3/library/trace}}
در پایتون\LTRfootnote{Python} یا کتابخانه جِی‌دی‌آی
\LTRfootnote{JDI (Java Debugger Interface) - \hyperlink{https://docs.oracle.com/javase/8/docs/jdk/api/jpda/jdi/}{docs.oracle.com/javase/8/docs/jdk/api/jpda/jdi}}
 در جاوا\LTRfootnote{Java} که مسیر اجرای کد سیستم تحت آزمون را می‌توان با این کتابخانه‌ها دنبال کرد. البته کتابخانه‌های ذکر شده تنها به‌عنوان مثال بیان شده‌اند و ممکن است در عمل از کتابخانه‌های دیگری نیز استفاده شود. هدف از بیان این مطلب این است که خودکارسازی کامل فرآیند تولید آزمایه در روش آزمون تصادفی تطبیقی مبتنی بر امتیازدهی غیرممکن نیست و با دنبال کردن خودکارسازی کامل، می‌توان یک ابزار برای این روش ارائه کرد.



