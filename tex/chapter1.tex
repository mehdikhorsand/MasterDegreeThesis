% !TeX root=../main.tex

\chapter{مقدمه}\label{chapter1}

\section{انگیزه پژوهش}
سیستم‌های نرم‌افزاری با قابلیت بالا \cite{dyson2004architecting}، مانند سامانه‌های پردازش تراکنش‌های مالی، هسته معاملات سهام و سوییچ‌های پرداخت الکترونیک، از جمله نرم‌افزارهایی هستند که اطمینان از درستی عملکرد آن‌ها همواره امری مهم و ضروری بوده است زیرا که وجود اشکال\LTRfootnote{Bug} در این‌گونه سیستم‌ها هرچند که خسارت جانی به دنبال نخواهد داشت، اما ممکن است منجر به خسارت‌های مالی بزرگ و جبران‌ناپذیری شود.  بنابراین، لازم است که این سیستم‌های نرم‌افزاری با انواع مختلف ورودی‌های ممکن برنامه مورد آزمایش قرار گیرند و عملکرد آن‌ها بررسی شود.

 یکی از ابتدایی‌ترین روش‌ها برای تولید انواع ورودی‌های ممکن برای این دسته از سیستم‌های نرم‌افزاری، استفاده از روش آزمون تصادفی
 \LTRfootnote{Random Testing}\cite{hamlet1994random}
 است. با استفاده از این روش و تولید ورودی‌های تصادفی برای این دسته از سیستم‌های نرم‌افزاری می‌توان از صحت عملکرد آن‌ها اطمینان حاصل کرد. روش آزمون تصادفی مزایایی مانند سادگی، خودکارسازی و پوشش گسترده با تولید مجموعه‌ای متنوع از ورودی‌های تصادفی را دارد، اما معایبی همچون ناکارآمدی، تمرکز محدود و از همه مهم‌تر تکرارپذیری باعث می‌شود که این روش برای آزمون این دسته از سیستم‌های نرم‌افزاری کافی نباشد.
اما روش آزمون تصادفی تطبیقی
\LTRfootnote{Adaptive Random Testing}\cite{huang2019survey}
 یکی از روش‌های آزمون نرم‌افزار به صورت خودکار است که با بهره‌گیری از ورودی‌های تصادفی سعی دارد آزمایه‌های پراکنده روی دامنه ورودی سیستم تحت آزمون تولید کند و کاستی‌های روش آزمون تصادفی، مانند ناکارآمدی، تمرکز محدود و تکرارپذیری را جبران کند.

در این پژوهش هدف ما استفاده از روش‌های آزمون تصادفی تطبیقی جهت تولید خودکار آزمایه\LTRfootnote{Testcase} برای این دسته از سیستم‌های نرم‌افزاری است و به‌عنوان نمونه از این گونه سیستم‌ها، یک مطالعه موردی روی هسته معاملات سهام\LTRfootnote{Matching Engine} انجام خواهد شد.

\section{چالش‌ها}

در این بخش، چالش‌هایی که در پیاده‌سازی روش آزمون تصادفی تطبیقی با آن‌ها مواجه خواهیم شد، به طور خلاصه لیست شده‌اند. این چالش‌ها همواره در ارائه هر استراتژی جدید برای روش آزمون تصادفی تطبیقی، مورد توجه قرار گرفته شده‌اند.

\begin{itemize}
	\item \textbf{تعریف معیار میزان تفاوت یا «فاصله» بین آزمایه‌ها}
	
در آزمون تصادفی تطبیقی، آزمایه‌ها هرچه بر روی دامنه ورودی پراکنده‌تر باشند، در کنار یکدیگر عملکرد بهتری برای تشخیص اشکالات سیستم تحت آزمون خواهند داشت. با توجه به این نکته، برای پیاده‌سازی آزمون تصادفی تطبیقی روی سیستم‌هایی با ورودی شئ‌گرا، باید معیار میزان تفاوت یا فاصله بین آزمایه‌ها تعریف شود تا با استفاده از فاصله بین آزمایه‌ها، بتوان آزمایه‌های پراکنده‌تری برای سیستم تحت آزمون انتخاب کرد و احتمال یافتن اشکال توسط آزمایه‌ها را افزایش داد.
	
	\item \textbf{هزینه انتخاب آزمایه در تعداد آزمایه بالا}
	
چالش بعدی هزینه اجرای روش آزمون تصادفی تطبیقی است. برای تولید یک مجموعه آزمایه که بر روی دامنه ورودی سیستم تحت آزمون به‌صورت پراکنده پخش شده‌اند، واضح است که باید تعداد زیادی فاصله بین آزمایه‌ها محاسبه شود. هر چه تعداد آزمایه‌ها بیشتر می‌شود، تعداد محاسبات فاصله لازم نیز هر بار بیشتر و بیشتر خواهد شد. در نتیجه، چالش مهم این است که چگونه می‌توان در طی رشد الگوریتم و در مراحل بالاتر، زمانی که تعداد آزمایه‌ها افزایش یافته است، تعداد محاسبات فاصله را کاهش داد.
	
	\item \textbf{جامعیت استراتژی ارائه‌شده}
	
چالش دیگری که در ارائه یک روش آزمون تصادفی تطبیقی وجود دارد، این است که روش ارائه شده باید قابل پیاده‌سازی روی هر سیستم تحت آزمونی باشد و به یک دسته خاص از سیستم‌ها محدود نشود و همچنین به‌جز کد سیستم تحت آزمون، اطلاعات دیگری را نیاز نداشته باشد.

\end{itemize}

در نتیجه، به‌طور کلی، مهم‌ترین چالش‌ها عبارتند از: ارائه یک استراتژی مناسب برای محاسبه میزان تفاوت بین آزمایه‌ها که علاوه بر کارآمدی، هزینه پیاده‌سازی کمی داشته باشد و بر روی همه سیستم‌های نرم‌افزاری قابلیت پیاده‌سازی داشته باشد.

\section{خلاصه روش‌شناسی}

در این پژوهش، روشی برای محاسبه میزان تفاوت بین آزمایه‌ها برای همه انواع سیستم‌های تحت آزمون با هر نوع دامنه ورودی ارائه شده است که به‌جای استفاده از «فاصله» به‌عنوان معیار تفاوت، از رویکرد امتیازدهی بهره می‌گیرد. در این روش، با توجه به مجموعه‌ رفتارهایی که آزمایه‌ها در حین اجرای خود از سیستم تحت آزمون مشاهده می‌کنند، به آزمایه‌های کاندید امتیاز داده می‌شود. هر چه این رفتارها توسط آزمایه‌های انتخاب‌شده قبلی کمتر مشاهده شده باشند، آزمایه کاندید برای مشاهده آن رفتار امتیاز بیشتری کسب می‌کند، و برعکس، هرچه این رفتارها بیشتر توسط آزمایه‌های انتخاب‌شده قبلی مشاهده شده باشند، آزمایه کاندید برای مشاهده آن رفتار امتیاز کمتری دریافت خواهد کرد. در پایان، مجموع امتیازات هر آزمایه کاندید براساس رفتارهای مختلفی که مشاهده کرده است، محاسبه شده و آزمایه‌ای انتخاب می‌شود که بیشترین مجموع امتیازات را کسب کرده باشد.

\section{خلاصه دستاوردها و نتایج}

روش ارائه‌شده در این پژوهش از نظر معیارهای ارزیابی میزان پوشش و معیارهای ارزیابی قدرت تشخیص اشکال، نسبت به مجموعه روش‌های انتخاب آزمایه با استفاده از محاسبه فاصله عملکرد بهتری از خود نشان داده است. این روش علاوه بر توانایی محاسبه میزان تفاوت بین دو آزمایه، قادر است میزان تفاوت بین یک آزمایه و یک مجموعه آزمایه و حتی میزان تفاوت دو مجموعه آزمایه مجزا را نیز محاسبه کند. در نتیجه، هزینه انتخاب آزمایه مبتنی بر امتیازدهی، برخلاف روش‌های انتخاب آزمایه مبتنی بر محاسبه فاصله، به تعداد آزمایه‌های انتخاب‌شده قبلی وابسته نیست.
همچنین، روش پیشنهادی انعطاف‌پذیری بالایی دارد و به آزمونگر این امکان را می‌دهد که معیارهای ارزیابی دلخواه خود برای امتیازدهی به مجموعه آزمایه‌های کاندید را با توجه به ویژگی‌های سیستم تحت آزمون تعریف کند.

\section{ساختار پایان‌نامه}

در فصل~\ref{chapter1}، خلاصه‌ای از پرسش پژوهش، اهمیت آن، مسیر پژوهش و نتایج به دست آمده بیان شده است و در فصل  \ref{chapter2} برخی مفاهیم اولیه توضیح داده‌ شده‌ است. پیشینه پژوهش و روال کلی انتخاب آزمایه با استفاده از محاسبه فاصله در فصل~\ref{chapter3} مورد بررسی قرار گرفته است. در فصل~\ref{chapter4}، روش پیشنهادی انتخاب آزمایه مبتنی بر امتیازدهی در آزمون تصادفی تطبیقی شرح داده شده است. در فصل~\ref{chapter5}، عملکرد استراتژی‌های انتخاب آزمایه مبتنی بر محاسبه فاصله با استراتژی‌های انتخاب آزمایه مبتنی بر امتیازدهی با توجه به معیارهای ارزیابی میزان پوشش و میزان قدرت تشخیص اشکال، مورد تحلیل و بررسی قرار گرفته است. سپس، بر اساس تحلیل‌های انجام‌شده، مزایای استراتژی انتخاب آزمایه مبتنی بر امتیازدهی نسبت به استراتژی انتخاب آزمایه مبتنی بر  محاسبه فاصله تشریح شده است. در فصل~\ref{chapter6} نیز خلاصه‌ای از نتایج به دست آمده ارائه شده است و چند نمونه از کارهایی که می‌توان برای گسترش استراتژی انتخاب آزمایه مبتنی بر امتیازدهی انجام داد، مثال زده شده است.












