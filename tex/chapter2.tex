\chapter{مفاهیم اولیه}
\label{chapter2}

\section{آزمون نرم‌افزار و اهمیت آن}

آزمون نرم‌افزار یکی از مراحل حیاتی در فرآیند توسعه نرم‌افزار است که به اطمینان از کیفیت و عملکرد صحیح نرم‌افزار در شرایط مختلف کمک می‌کند. اهمیت آزمون نرم‌افزار از چند جنبه قابل بررسی است:

\begin{itemize}
	\item \textbf{کشف اشکالات نرم‌افزاری}
	
	آزمون نرم‌افزار به شناسایی و رفع اشکالات نرم‌افزاری قبل از اینکه نرم‌افزار به دست کاربران برسد، کمک می‌کند. این کار از بروز مشکلات جدی و نارضایتی کاربران جلوگیری می‌کند.
	\item \textbf{افزایش اعتماد به کیفیت نرم‌افزار}
	
	با انجام آزمون‌های دقیق و کامل، می‌توان اطمینان حاصل کرد که نرم‌افزار به درستی و با کیفیت بالا کار می‌کند. این اعتماد برای مشتریان و کاربران نهایی بسیار مهم است.
	\item \textbf{کاهش هزینه‌ها}
	
	شناسایی و رفع مشکلات در مراحل اولیه توسعه نرم‌افزار بسیار کم‌هزینه‌تر از رفع آنها پس از انتشار نرم‌افزار است. آزمون مؤثر نرم‌افزار می‌تواند هزینه‌های نگهداری و پشتیبانی را به میزان قابل توجهی کاهش دهد.
	\item \textbf{اطمینان از عملکرد صحیح نرم‌افزار در شرایط مختلف}
	
	آزمون نرم‌افزار تضمین می‌کند که نرم‌افزار در شرایط مختلف (مانند سیستم‌عامل‌های متفاوت، نسخه‌های مختلف مرورگرها و غیره) به درستی عمل می‌کند.
	\item \textbf{مستندسازی و درک بهتر سیستم}
	
	فرآیند آزمون شامل نوشتن آزمایه‌ها و مستندسازی آن‌ها است که به توسعه‌دهندگان کمک می‌کند تا درک بهتری از سیستم داشته باشند و در آینده به راحتی بتوانند تغییرات مورد نیاز را اعمال کنند.
\end{itemize}

\section{آزمون جعبه ‌سیاه}

آزمون جعبه سیاه
\LTRfootnote{Black Box Testing}\cite{nidhra2012black}
 یک روش آزمون نرم‌افزار است که در آن عملکرد سیستم بدون توجه به ساختار داخلی یا کد آن ارزیابی می‌شود. در این روش، آزمون‌گر تنها از طریق ورودی‌ها و خروجی‌ها با سیستم تعامل می‌کند و بررسی می‌کند که آیا سیستم مطابق با نیازها و مشخصات تعیین‌شده عمل می‌کند یا خیر. این روش بر روی عملکرد سیستم تمرکز دارد و بررسی می‌کند که سیستم چه کاری انجام می‌دهد، نه اینکه چگونه این کار را انجام می‌دهد. این روش برای آزمون قابلیت‌های نرم‌افزار از دیدگاه کاربر نهایی بسیار مناسب است.


\section{تولید آزمایه به صورت خودکار}
تولید خودکار آزمایه 
\LTRfootnote{Testcase}
به معنای ایجاد سناریوهای آزمون به صورت خودکار، بدون نیاز به دخالت دستی است~\cite{shanthi2011automated}. در این روش‌ها آزمونگر از تکنیک‌ها و ابزارهای ویژه‌ای استفاده می‌کند تا سناریوهای متنوعی برای بررسی عملکرد سیستم تحت آزمون
\LTRfootnote{System Under Test}
ایجاد کند. مزایای تولید آزمایه به صورت خودکار عبارت‌اند از:
\begin{itemize}
	\item \textbf{پوشش گسترده‌تر آزمون}
	
	تولید خودکار آزمایه‌ها، مواردی را پوشش می‌دهد که ممکن است در آزمون‌های دستی نادیده گرفته شوند. این امر به ویژه در آزمون نرم‌افزارهای پیچیده و بزرگ اهمیت بیشتری دارد.
	\item \textbf{افزایش کارایی آزمون}
	
	با خودکارسازی فرآیند تولید آزمایه‌ها، می‌توان تعداد زیادی آزمایه در مدت زمان کوتاهی ایجاد کرد که این ویژگی در فازهای توسعه و نگهداری نرم‌افزار بسیار مفید است.
	\item \textbf{کشف اشکالات ناشناخته}
	
	تولید خودکار آزمایه‌ها به شناسایی اشکالاتی کمک می‌کند که ممکن است در سناریوهای پیش‌بینی‌نشده مشاهده شوند. آزمون‌های تصادفی به خصوص در این زمینه بسیار مؤثر هستند.
	\item \textbf{کاهش دخالت انسانی}
	
	خودکارسازی تولید آزمایه‌ها باعث کاهش خطاهای انسانی می‌شود و دقت و اطمینان آزمون‌ها را افزایش می‌دهد.
\end{itemize}

\section{آزمون تصادفی}

از ساده‌ترین و ابتدایی‌ترین روش‌های جعبه ‌سیاه که در آن تولید آزمایه به صورت خودکار انجام می‌شود می‌توان به روش آزمون تصادفی
\LTRfootnote{Random Testing}\cite{hamlet1994random}
 اشاره کرد. در این روش بعد از تعریف دامنه ورودیِ سیستم تحت آزمون، مجموعه‌ای از مقادیر ورودی به صورت تصادفی در داخل فضای ورودی تعریف شده، تولید می‌شود. به این ترتیب، عملکرد سیستم تحت آزمون در مواجهه با انواع مختلف سناریوهای ورودی، آزمایش می‌شود.

\section{آزمون تصادفی تطبیقی}

آزمون تصادفی تطبیقی
\LTRfootnote{Adaptive Random Testing}\cite{chen2005adaptive}
 یک روش بهبود یافته از آزمون تصادفی است که با هدف توزیع یکنواخت‌تر آزمایه‌ها در فضای ورودی انجام می‌شود. در این روش، به جای انتخاب تصادفی کامل، تلاش می‌شود آزمایه‌ها به شکلی انتخاب شوند که به طور مساوی در سراسر فضای ورودی پراکنده باشند. این کار احتمال کشف اشکالات نرم‌افزار را افزایش می‌دهد و باعث می‌شود آزمون‌ها با کارایی بیشتری نسبت به آزمون تصادفی معمولی انجام شوند.

آز آنجایی که در روش آزمون تصادفی تطبیقی می‌توان به روش‌های مختلفی معیار میزان تفاوت بین آزمایه‌ها را تعیین کرد و در بعضی روش‌های آزمون تصادفی تطبیقی با توجه به معیار‌های پوشش درون کد سیستم تحت آزمون، میزان تفاوت بین آزمایه‌ها محاسبه می‌شود
\cite{zhou2010using}\cite{chen2021novel}
؛ نمی‌توان آزمون تصادفی تطبیقی را در دسته روش‌های آزمون جعبه سیاه قرار داد. اما در بعضی از پیاده‌سازی‌های آزمون تصادفی تطبیقی، میزان تفاوت بین آزمایه‌ها با توجه به میزان تفاوت بین ورودی‌ها بدون در نظر گرفتن سیستم تحت آزمون محاسبه می‌شود
\cite{ciupa2008artoo}\cite{chen2007test}
 که این دسته از پیاده‌سازی‌های آزمون تصادفی تطبیقی می‌توانند در درسته روش‌های آزمون جعبه سیاه قرار گیرند.

\section{روش تقسیم‌بندی فضای ورودی}

از جمله روش‌های دیگری که سعی بر تولید آزمایه‌های متفاوت و پراکنده روی دامنه ورودی سیستم تحت آزمون دارد؛ روش تقسیم‌بندی فضای ورودی
\LTRfootnote{Input Space(Domain) Partitioning (ISP)}\cite{vagoun1996input}
 است.
روش تقسیم‌بندی فضای ورودی یک روش آزمون نرم‌افزار به صورت دستی است که در آن فضای ورودی به چندین بخش\LTRfootnote{Partition} تقسیم می‌شود. هر بخش شامل مجموعه‌ای از ورودی‌های مشابه است که انتظار می‌رود رفتار سیستم برای آن‌ها یکسان باشد. سپس، آزمون‌گر از هر بخش یک یا چند ورودی انتخاب کرده و سیستم را با آن‌ها آزمون می‌کند. این روش به بهبود پوشش آزمون و کاهش تعداد آزمایه‌ها کمک می‌کند.

\section{پوشش شاخه}

پوشش شاخه‌
\LTRfootnote{Branch Coverage}\cite{wei2012branch}
 یکی از معیارهای ارزیابی در آزمون نرم‌افزار است که هدف آن بررسی این است که آیا تمام شاخه‌های ممکن در یک برنامه طی فرآیند آزمون حداقل یک بار اجرا شده‌اند یا خیر. در این روش، هر نقطه تصمیم‌گیری (مانند شرط‌ها یا حلقه‌ها) بررسی می‌شود تا اطمینان حاصل شود که هر دو مسیر «درست» و «نادرست» آن‌ها آزمایش شده است. این معیار ارزیابی کمک می‌کند تا شاخه‌هایی از کد که ممکن است باعث مشاهده اشکال در سیستم تحت آزمون شوند، شناسایی شوند و بدین ترتیب پوشش کامل‌تری روی کد سیستم تحت آزمون حاصل شود.

\section{آزمون جهش}

آزمون جهش\LTRfootnote{Mutation Testing} یک روش پیشرفته آزمون نرم‌افزار است که با ایجاد تغییرات جزئی و عمدی به نام «جهش» در کد منبع برنامه، عملکرد آزمایه‌ها را ارزیابی می‌کند \cite{parsai2020comparing}. هدف آزمون جهش این است که بررسی شود آیا آزمایه‌های موجود قادر به شناسایی این تغییرات هستند یا خیر. اگر آزمایه‌ها نتوانند این جهش‌های ایجاد شده را شناسایی کنند، نشان‌دهنده ضعف مجموعه آزمایه‌ها است. این روش به بهبود کیفیت آزمایه‌ها و شناسایی بخش‌هایی از کد که به‌خوبی پوشش داده نشده‌اند، کمک می‌کند.

