% !TeX root=../main.tex
\chapter{نتایج و ارزیابی}\label{chapter4}

برای ارزیابی روش‌های آزمون تصادفی تطبیقی مبتنی بر استراتژی امتیازدهی که در این پژوهش ارائه شده است، روش‌های \lr{ART\_AutoISP} و \lr{ART\_AutoISP\_C} به عنوان نمایندگان این دسته از روش‌ها، با روش‌های \lr{ART\_WTClustering} و \lr{ART\_TFCClustering} که در آخرین پژوهش‌های انجام‌شده در حوزه روش‌های آزمون تصادفی تطبیقی معرفی شده‌اند و طبق مستندات این مقاله از همه روش‌های ارائه‌شده تاکنون عملکرد نسبتاً بهتری دارند و همچنین روش آزمون تصادفی سنتی، مقایسه خواهند شد. نتایج این مقایسه بر اساس معیارهای ارزیابی که در ادامه توضیح داده شده‌اند، تحلیل و بررسی خواهند شد.

\section{معیارهای ارزیابی}

در این پژوهش از معیارهای ارزیابی پوشش و معیارهای ارزیابی قدرت تشخیص خطا مختص روش‌های آزمون تصادفی تطبیقی جهت ارزیابی و مقایسه نتیجه با پژوهش‌های پیشین استفاده شده است.

\subsection{معیارهای میزان پوشش}

در این پژوهش، از معیارهای پوشش شاخه و پوشش جهش برای ارزیابی استراتژی‌های انتخاب آزمایه مبتنی بر فاصله و انتخاب آزمایه مبتنی بر امتیازدهی استفاده اشده است.

\subsubsection{پوشش شاخه}

پوشش شاخه \LTRfootnote{Branch Coverage} یکی از معیارهای کلیدی در ارزیابی جامعیت آزمایش‌های نرم‌افزاری است که بر اساس بررسی مسیرهای مختلف اجرای کد تعیین می‌شود. این معیار با هدف اطمینان از آزمایش همه مسیرهای ممکن که از تصمیمات منطقی کد نشأت می‌گیرند، استفاده می‌شود. به عبارت دیگر، پوشش شاخه میزان بررسی و آزمایش تمامی انشعابات (شرایط) موجود در کد را اندازه‌گیری می‌کند.
\begin{itemize}
	
	\item \textbf{فرآیند ارزیابی پوشش شاخه}

فرآیند ارزیابی پوشش شاخه شامل مراحل زیر است:

\begin{enumerate}
	\item \textbf{شناسایی شاخه‌ها:} در این مرحله، تمامی نقاط تصمیم‌گیری در کد (مانند دستورات شرطی \texttt{if-else}، \texttt{switch-case} و حلقه‌ها) شناسایی می‌شوند. هر یک از این نقاط می‌تواند به دو یا چند شاخه ممکن منتهی شود که باید توسط آزمایه‌ها پوشش داده شوند.
	\item \textbf{اجرای آزمایه‌ها:} مجموعه‌ای از آزمایه‌ها طراحی و اجرا می‌شوند تا تمام شاخه‌های شناسایی‌شده را پوشش دهند. هدف این است که برای هر تصمیم‌گیری در کد، تمام شاخه‌های ممکن (مسیرهای \textit{true} و \textit{false}) بررسی شوند.
	\item \textbf{محاسبه پوشش شاخه:} پس از اجرای آزمایه‌ها، میزان پوشش شاخه با محاسبه نسبت شاخه‌های پوشش‌داده‌شده به کل شاخه‌های موجود در کد به دست می‌آید. این معیار معمولاً به صورت درصدی بیان می‌شود:
	\[
	\text{پوشش شاخه (\%)} = \frac{\text{تعداد شاخه‌های اجرا شده}}{\text{تعداد کل شاخه‌ها}} \times 100
	\]
\end{enumerate}

\item \textbf{اهمیت پوشش شاخه}

پوشش شاخه یک معیار قوی‌تر و دقیق‌تر از پوشش خط\LTRfootnote{Line Coverage} محسوب می‌شود، زیرا علاوه بر پوشش دادن تمام خطوط کد، اطمینان حاصل می‌کند که تمامی شاخه‌های ممکن از تصمیمات منطقی کد نیز مورد آزمایش قرار گرفته‌اند. این معیار به ویژه در شناسایی نقاطی از کد که ممکن است در صورت آزمایش نشدن به بروز خطاهای پنهان منجر شوند، اهمیت دارد. پوشش شاخه به توسعه‌دهندگان کمک می‌کند تا با اطمینان بیشتری درستی و کامل بودن آزمایه‌های خود را ارزیابی کنند و نقاط ضعف احتمالی را شناسایی و برطرف کنند.
\end{itemize}

\subsubsection{پوشش جهش}

یکی از معیارهای مهم در ارزیابی کیفیت و اثربخشی آزمون‌های نرم‌افزاری، معیار پوشش جهش\LTRfootnote{Mutation Coverage} است. این معیار به بررسی توانایی آزمایه‌ها در شناسایی تغییرات عمدی و کوچک (جهش‌ها) در کد منبع نرم‌افزار می‌پردازد. ایده اصلی پشت این معیار، این است که با ایجاد تغییرات جزئی در کد منبع، بتوان قابلیت کشف خطاها توسط مجموعه آزمایه‌ها را ارزیابی کرد.

\begin{itemize}
	\item \textbf{فرآیند ارزیابی پوشش جهش}

فرآیند ارزیابی پوشش جهش شامل مراحل زیر است:

\begin{enumerate}
	\item \textbf{ایجاد جهش‌ها:} در این مرحله، تغییرات کوچکی در کد منبع نرم‌افزار ایجاد می‌شود. این تغییرات به صورت عمدی و هدفمند اعمال می‌شوند تا به اصطلاح «کد جهش‌یافته» تولید شود. برای مثال، ممکن است یک عملگر منطقی مانند \texttt{==} به \texttt{!=} تغییر داده شود.
	\item \textbf{اجرای آزمایه‌ها:} مجموعه آزمایه‌های موجود روی کد جهش‌یافته اجرا می‌شوند. هدف این است که آزمایه‌ها بتوانند این جهش‌های عمدی را شناسایی کرده و با شکست مواجه شوند، یعنی خروجی آزمایه‌ها پس از اعمال جهش با خروجی اصلی متفاوت باشد.
	\item \textbf{محاسبه پوشش جهش:} اگر آزمایه‌ها بتوانند جهش‌ها را شناسایی کنند و خروجی‌های نادرست ایجاد شده توسط کد جهش‌یافته را تشخیص دهند، گفته می‌شود که آزمایه موفق به «کشتن» آن جهش شده است. میزان پوشش جهش به صورت درصدی از جهش‌های کشته‌شده به کل جهش‌های اعمال‌شده محاسبه می‌شود:
	\[
	\text{پوشش جهش (\%)} = \frac{\text{تعداد جهش‌های کشته‌شده توسط آزمایه‌ها}}{\text{تعداد کل جهش‌ها}} \times 100
	\]
\end{enumerate}

	\item \textbf{اهمیت پوشش جهش}

پوشش جهش به عنوان یک معیار مکمل برای سایر معیارهای پوشش کد، مانند پوشش خط یا پوشش شاخه\LTRfootnote{Branch Coverage}، مورد استفاده قرار می‌گیرد. این معیار می‌تواند به شناسایی ضعف‌های موجود در مجموعه آزمایه‌ها کمک کند و نقاطی از کد را که به درستی آزمون نشده‌اند، نمایان سازد. در نهایت، پوشش جهش کمک می‌کند تا مطمئن شویم که آزمایه‌های طراحی‌شده به اندازه کافی جامع هستند و می‌توانند تغییرات نامطلوب و خطاهای پنهان را در کد نرم‌افزار کشف کنند.

\end{itemize}

\subsection{معیارهای میزان قدرت تشخیص خطا}
معیارهای ارزیابی \lr{F-measure} و \lr{F-time} به‌طور گسترده برای تحلیل قدرت تشخیص خطای روش‌های آزمون تصادفی تطبیقی در پژوهش‌های متعددی که تاکنون در این زمینه ارائه شده‌اند، مورد استفاده قرار گرفته‌اند. هر چه مقدار این معیارها در یک روش آزمون تصادفی تطبیقی کمتر باشد، نشان‌دهنده قابلیت بالاتر آن روش در تشخیص خطاها است.

\subsubsection{معیار ارزیابی \lr{F-measure}}

در آزمون تصادفی تطبیقی، معیار \lr{F-Measure} به مدت زمانی اشاره دارد که طول می‌کشد تا اولین خطا در سیستم تحت آزمون کشف شود. این معیار به‌طور خاص بر ارزیابی کارایی روش آزمون تمرکز دارد، به این صورت که مشخص می‌کند روش به‌کاررفته چه تعداد آزمایه نیاز دارد که تولید کند تا بتواند اولین خطا را در سیستم تحت آزمون شناسایی کند. F-Measure به عنوان یک شاخص کلیدی در ارزیابی روش‌های آزمون تصادفی تطبیقی به کار می‌رود. روش‌های آزمونی که دارای مقدار \lr{F-Measure} کوچکتری هستند، به عنوان روش‌های کارآمدتر شناخته می‌شوند زیرا قادر به کشف خطاها در تعداد آزمایه کمتری بوده‌اند.

\subsubsection{معیار ارزیابی F-Time}

\lr{F-Time} نیز یکی دیگر از معیارهای مهم در ارزیابی روش‌های آزمون تصادفی تطبیقی است. این معیار به زمان لازم برای کشف اولین خطا در سیستم تحت آزمون اشاره دارد. \lr{F-Time} به‌طور دقیق‌تر نشان می‌دهد که روش آزمون چه مدت زمانی نیاز دارد تا اولین خطا را در شرایط واقعی کشف کند.

این معیار به‌عنوان معیاری برای اندازه‌گیری سرعت عملکرد روش‌های آزمون تصادفی تطبیقی به کار می‌رود. روش‌هایی که زمان کمتری برای کشف اولین خطا نیاز دارند، معمولاً کارآمدتر محسوب می‌شوند.

\section{منصفانه بودن مقایسه استراتژی‌ها}

در این پژوهش تلاش شده است که مقایسه بین استراتژی‌ها به صورت منصفانه انجام شود. یکی از اقداماتی که برای تحقق این هدف انجام شده، این است که آزمایه‌های کاندید ورودی استراتژی‌های انتخاب آزمایه در هر مرحله یکسان بوده است. در واقع، این استراتژی‌های مختلف به صورت موازی با یکدیگر اجرا شده‌اند و تمامی این روش‌ها مرحله به مرحله با هم پیش رفته‌اند و در هر مرحله از یک مجموعه کاندید یکسان، آزمایه بعدی خود را انتخاب کرده‌اند.

همچنین، به ازای هر معیار ارزیابی، هر کدام از استراتژی‌های مورد نظر چندین بار مورد ارزیابی قرار گرفته‌اند و میانگین نتیجه ارزیابی آن‌ها با یکدیگر مقایسه شده است.

\section{معرفی سیستم‌های تحت آزمون}

در این پژوهش از دو سیستم تحت آزمون برای ارزیابی استراتژی‌های انتخاب آزمایه بر اساس فاصله و انتخاب آزمایه بر اساس امتیاز در روش آزمون تصادفی تطبیقی استفاده شده است که در ادامه به صورت مختصر این سیستم‌های تحت آزمون معرفی شده‌اند.

\subsection{سیستم پردازش تراکنش‌های مالی}

سیستم تحت آزمون اول یک نمونه از سیستم‌های پردازش تراکنش مالی\LTRfootnote{Matching Engine} است که وظیفه پردازش انواع مختلف سفارش‌های خرید و فروش را دارد. ساختار هر آزمایه برای این سیستم به این صورت است که در هر آزمایه ابتدا به صورت تصادفی چند کارگزار و چند سهامدار مشخص می‌شوند و سپس یک سری سفارش از انواع مختلف به صورت تصادفی تولید می‌شوند. خروجی هر آزمایه وضعیت کارگزاران و سهامداران بعد از انجام سفارشات و همچنین معاملات انجام‌شده پس از اضافه شدن هر سفارش و وضعیت صف سفارشات بعد از هر اضافه شدن سفارش را نمایش می‌دهد. 

برای پی بردن به درستی یا نادرستی خروجی هر آزمایه بعد از اجرا، از کد هسکل\LTRfootnote{Haskell} سیستم مذکور به عنوان اوراکل\LTRfootnote{Oracle} استفاده شده است. خروجی هر آزمایه روی سیستم تحت آزمون با خروجی همان آزمایه روی کد هسکل مقایسه می‌شود و در صورت تفاوت خروجی‌ها، این تفاوت به معنای مشاهده شکست درون سیستم تحت آزمون تلقی خواهد شد.

\subsection{سیستم آسانسور}

سیستم تحت آزمون بعدی برنامه‌ای است که وظیفه کنترل آسانسورهای یک ساختمان را دارد. ساختار هر آزمایه برای این سیستم تحت آزمون به این صورت است که ابتدا مجموعه آسانسورها و تعداد طبقات یک ساختمان تعریف می‌شود و سپس از طبقات مختلف درخواست آسانسور به سمت‌های بالا یا پایین داده می‌شود و همچنین درخواست برای طبقات مختلف درون آسانسورها ارائه می‌شود. سپس در خروجی این برنامه، مجموعه عملیات‌های انجام‌شده توسط آسانسورها پس از هر درخواست در زمان‌های مختلف مشخص می‌شود.

از یک نسخه ویرایش‌نشده این برنامه به عنوان نسخه اوراکل استفاده شده است و برای سیستم تحت آزمون، چند تغییر در قسمت‌های مختلف برنامه اصلی اعمال شده تا کد سیستم دارای خطا باشد. برای بررسی درستی عملکرد سیستم تحت آزمون، خروجی‌های اوراکل و سیستم تحت آزمون به ازای هر آزمایه با یکدیگر مقایسه می‌شوند. عدم تطابق خروجی‌ها به معنای مشاهده شکست درون سیستم تحت آزمون است.

\section{ارزیابی استراتژی‌ها با استفاده از معیار میزان پوشش}

در این بخش، استراتژی‌های انتخاب آزمایه بر اساس فاصله و انتخاب آزمایه بر اساس امتیاز روی سیستم‌های تحت آزمون ذکر شده، با توجه به معیارهای ارزیابی میزان پوشش، با یکدیگر مقایسه شده‌اند. 

برای اینکه بتوانیم استراتژی‌های مذکور را با استفاده از معیار ارزیابی میزان پوشش ارزیابی کنیم، باید سیستم تحت آزمون بدون خطا باشد تا همه آزمایه‌ها بدون مشکل قبول\LTRfootnote{Pass} شوند. در نتیجه، در روال ارزیابی استراتژی‌ها با توجه به معیارهای میزان پوشش، مقایسه‌ای بین خروجی سیستم تحت آزمون و خروجی اوراکل برنامه صورت نگرفته است.

\subsection{میزان پوشش روی برنامه پردازش تراکنش‌های مالی}

در این بخش، به ازای استراتژی‌های مختلف انتخاب آزمایه بر اساس فاصله و استراتژی‌های انتخاب آزمایه بر اساس امتیاز برای سیستم پردازش تراکنش‌های مالی، تعداد ۲۰ آزمایه در هر روش تولید شده است و میزان پوشش مجموعه‌های آزمایه‌ها روی سیستم پردازش تراکنش‌های مالی سنجیده شده است. سپس این فرآیند ۱۰۰ بار تکرار شده است و میانگین معیار میزان پوشش در این ۱۰۰ بار به ازای هر روش به دست آمده و با میانگین میزان پوشش سایر روش‌ها مقایسه شده است.

\begin{table}[H]
	\centering
	\begin{LTR}
		\begin{tabular}{
				|>{\centering\arraybackslash\footnotesize}m{3cm}|
				>{\centering\arraybackslash\footnotesize}m{3cm}|
				>{\centering\arraybackslash\footnotesize}m{3.5cm}|
				>{\centering\arraybackslash\footnotesize}m{3.5cm}|
			}
			\hline
			\textbf{\rl{استراتژی}} & \textbf{\rl{روش}} & \textbf{\rl{میانگین پوشش شاخه}} & \textbf{\rl{میانگین پوشش جهش}} \\ \hline
			\rl{تصادفی سنتی} & \lr{RT} & \lr{67.3}\% &  \lr{42.8}\% \\ \hline
			\multirow{2}{*}{\rl{مبتنی بر فاصله}} & \lr{ART\_WTClustering} & \lr{75.1}\% &  \lr{54.3}\% \\ \cline{2-4} 
			& \lr{ART\_TFClustering} & \lr{78.4}\% & \lr{57.1}\% \\ \hline
			\multirow{2}{*}{\rl{مبتنی بر امتیاز}} & \lr{ART\_AutoISP} & \lr{86.6}\% &  \lr{61.7}\% \\ \cline{2-4} 
			& \lr{ART\_AutoISP\_C} & \lr{87.1}\% & \lr{62}\% \\ \hline
		\end{tabular}
	\end{LTR}
	\caption{\footnotesize میانگین میزان پوشش روی برنامه پردازش تراکنش‌های مالی}
\end{table}

\subsection{میزان پوشش روی برنامه آسانسور}

همانند سیستم تحت آزمون قبلی، در این بخش نیز به ازای استراتژی‌های مختلف انتخاب آزمایه بر اساس فاصله و استراتژی‌های انتخاب آزمایه بر اساس امتیاز برای سیستم آسانسور، تعداد 10 آزمایه در هر روش تولید شده است و میزان پوشش مجموعه‌های آزمایه‌ها روی سیستم آسانسور سنجیده شده است. سپس این فرآیند 50 بار تکرار شده است و میانگین معیار میزان پوشش در این 50 بار به ازای هر روش به دست آمده و با میانگین میزان پوشش سایر روش‌ها مقایسه شده است.

\begin{table}[H]
	\centering
	\begin{LTR}
		\begin{tabular}{
				|>{\centering\arraybackslash\footnotesize}m{3cm}|
				>{\centering\arraybackslash\footnotesize}m{3cm}|
				>{\centering\arraybackslash\footnotesize}m{3.5cm}|
				>{\centering\arraybackslash\footnotesize}m{3.5cm}|
			}
			\hline
			\textbf{\rl{استراتژی}} & \textbf{\rl{روش}} & \textbf{\rl{میانگین پوشش شاخه}} & \textbf{\rl{میانگین پوشش جهش}} \\ \hline
			\rl{تصادفی سنتی} & \lr{RT} & \lr{58.5}\% &  \lr{34.2}\% \\ \hline
			\multirow{2}{*}{\rl{مبتنی بر فاصله}} & \lr{ART\_WTClustering} & \lr{67.6}\% &  \lr{41.4}\% \\ \cline{2-4}
			& \lr{ART\_TFClustering} & \lr{70.6}\% & \lr{43.9}\% \\ \hline
			\multirow{2}{*}{\rl{مبتنی بر امتیاز}} & \lr{ART\_AutoISP} & \lr{79.5}\% &  \lr{49.8}\% \\ \cline{2-4}
			& \lr{ART\_AutoISP\_C} & \lr{81.1}\% & \lr{49.9}\% \\ \hline
		\end{tabular}
	\end{LTR}
	\caption{\footnotesize میانگین میزان پوشش روی برنامه آسانسور}
\end{table}

\subsection{تحلیل نتایج}

همانطور که مشاهده می‌کنید، با توجه به جدول‌های ۴.۲ و ۴.۱، استراتژی‌های انتخاب آزمایه مبتنی بر امتیازدهی نسبت به استراتژی‌های انتخاب آزمایه مبتنی بر فاصله بر روی سیستم‌های تحت آزمون تعریف‌شده، با توجه به معیارهای ارزیابی پوشش شاخه و پوشش جهش، عملکرد بهتری از خود نشان داده‌اند.

\section{ارزیابی استراتژی‌ها با استفاده از معیار میزان قدرت تشخیص خطا}

در این بخش، استراتژی‌های انتخاب آزمایه بر اساس فاصله و انتخاب آزمایه بر اساس امتیاز روی سیستم‌های تحت آزمون ذکر شده، با توجه به معیارهای ارزیابی میزان قدرت تشخیص خطا، با یکدیگر مقایسه شده‌اند. 

برای اینکه بتوانیم استراتژی‌های مذکور را با استفاده از معیار ارزیابی میزان قدرت تشخیص خطا ارزیابی کنیم، باید سیستم تحت آزمون خطادار باشد تا بتوان میزان قدرت تشخیص خطای مجموعه آزمایه‌ها را سنجید. در نتیجه، در روال ارزیابی استراتژی‌ها با توجه به معیارهای میزان قدرت تشخیص خطا، چند جهش در برنامه تحت آزمون ایجاد شده است که در برخی حالت‌های خاص ورودی در آزمایه‌ها باعث ایجاد تفاوت در خروجی برنامه تحت آزمون و خروجی اوراکل خواهد شد که به معنی شکست آزمایه و به عبارت دیگر تشخیص خطا در برنامه تحت آزمون خواهد بود.

\subsection{میزان قدرت تشخیص خطا روی برنامه پردازش تراکنش‌های مالی}

در این بخش، به ازای استراتژی‌های مختلف انتخاب آزمایه بر اساس فاصله و استراتژی‌های انتخاب آزمایه بر اساس امتیاز برای سیستم پردازش تراکنش‌های مالی، با توجه به تعریف معیارهای ارزیابی میزان قدرت تشخیص خطا، تعدادی آزمایه تولید شده است و معیارهای ارزیابی میزان قدرت تشخیص خطا سنجیده شده است. سپس این فرآیند ۱۰۰ بار تکرار شده است و میانگین معیار میزان قدرت تشخیص خطا در این ۱۰۰ بار به ازای هر روش به دست آمده و با میانگین میزان قدرت تشخیص خطا سایر روش‌ها مقایسه شده است.

\begin{table}[H]
	\centering
	\begin{LTR}
		\begin{tabular}{
				|>{\centering\arraybackslash\footnotesize}m{3cm}|
				>{\centering\arraybackslash\footnotesize}m{3cm}|
				>{\centering\arraybackslash\footnotesize}m{3.5cm}|
				>{\centering\arraybackslash\footnotesize}m{3.5cm}|
			}
			\hline
			\textbf{\rl{استراتژی}} & \textbf{\rl{روش}} & \textbf{\rl{میانگین \lr{F\_measure}}} & \textbf{\rl{میانگین \lr{F\_time (ms)}}} \\ \hline
			\rl{تصادفی سنتی} & \lr{RT} & \lr{56.1} &  \lr{4268} \\ \hline
			\multirow{2}{*}{\rl{مبتنی بر فاصله}} & \lr{ART\_WTClustering} & \lr{22.4} &  \lr{3095} \\ \cline{2-4} 
			& \lr{ART\_TFClustering} & \lr{19.2} &  \lr{2906} \\ \hline
			\multirow{2}{*}{\rl{مبتنی بر امتیاز}} & \lr{ART\_AutoISP} & \lr{16.3} &  \lr{2766} \\ \cline{2-4} 
			& \lr{ART\_AutoISP\_C} & \lr{13.7} &  \lr{2451} \\  \hline
		\end{tabular}
	\end{LTR}
	\caption{\footnotesize میانگین میزان قدرت تشخیص خطا روی برنامه پردازش تراکنش‌های مالی}
\end{table}

\subsection{میزان قدرت تشخیص خطا روی برنامه آسانسور}

همانند سیستم تحت آزمون قبلی، در این بخش به ازای استراتژی‌های مختلف انتخاب آزمایه بر اساس فاصله و استراتژی‌های انتخاب آزمایه بر اساس امتیاز برای سیستم آسانسور، با توجه به تعریف معیارهای ارزیابی میزان قدرت تشخیص خطا، تعدادی آزمایه تولید شده است و معیارهای ارزیابی میزان قدرت تشخیص خطا سنجیده شده است. سپس این فرآیند ۱۰۰ بار تکرار شده است و میانگین معیار میزان قدرت تشخیص خطا در این ۱۰۰ بار به ازای هر روش به دست آمده و با میانگین میزان قدرت تشخیص خطا سایر روش‌ها مقایسه شده است.

\begin{table}[H]
	\centering
	\begin{LTR}
		\begin{tabular}{
				|>{\centering\arraybackslash\footnotesize}m{3cm}|
				>{\centering\arraybackslash\footnotesize}m{3cm}|
				>{\centering\arraybackslash\footnotesize}m{3.5cm}|
				>{\centering\arraybackslash\footnotesize}m{3.5cm}|
			}
			\hline
			\textbf{\rl{استراتژی}} & \textbf{\rl{روش}} & \textbf{\rl{میانگین \lr{F\_measure}}} & \textbf{\rl{میانگین \lr{F\_time (ms)}}} \\ \hline
			\rl{تصادفی سنتی} & \lr{RT} & \lr{68.3} &  \lr{97465} \\ \hline
			\multirow{2}{*}{\rl{مبتنی بر فاصله}} & \lr{ART\_WTClustering} & \lr{33.5} &  \lr{61753} \\ \cline{2-4} 
			& \lr{ART\_TFClustering} & \lr{32} &  \lr{61245} \\ \hline
			\multirow{2}{*}{\rl{مبتنی بر امتیاز}} & \lr{ART\_AutoISP} & \lr{24.3} &  \lr{58429} \\ \cline{2-4} 
			& \lr{ART\_AutoISP\_C} & \lr{19.2} &  \lr{56440} \\  \hline
		\end{tabular}
	\end{LTR}
	\caption{\footnotesize میانگین میزان قدرت تشخیص خطا روی برنامه آسانسور}
\end{table}

\subsection{تحلیل نتایج}

با توجه به تعریف معیارهای ارزیابی میزان قدرت تشخیص خطا، \lr{F-measure} تعداد آزمایه‌های انتخاب‌شده تا رسیدن به اولین خطا درون سیستم تحت آزمون و \lr{F-time} مدت زمان صرف‌شده برای انتخاب و اجرای آزمایه‌های انتخاب‌شده است. در نتیجه، در مقایسه این دو معیار بر روی دو روش آزمون تصادفی تطبیقی، هر چه مقدار \lr{F-measure} و \lr{F-time} یک روش نسبت به دیگری کمتر باشد، به معنای قدرت بیشتر تشخیص خطا در آن روش نسبت به روش دیگر است.

با توجه به این توضیحات و نتایج ذکرشده در جدول‌های ۴.۳ و ۴.۴، استراتژی‌های انتخاب آزمایه مبتنی بر امتیازدهی نسبت به استراتژی‌های انتخاب آزمایه مبتنی بر فاصله بر روی سیستم‌های تحت آزمون تعریف‌شده، با توجه به معیارهای میزان قدرت تشخیص خطا، عملکرد بهتری از خود نشان داده‌اند.

\section{مقایسه هزینه فضایی و زمانی}

\subsection{هزینه فضایی و زمانی استراتژی‌های انتخاب آزمایه مبتنی بر محاسبه فاصله}

در روش‌های انتخاب آزمایه با استفاده از محاسبه فاصله، هر آزمایه کاندید با مجموعه آزمایه‌های انتخاب‌شده در مراحل قبلی آزمون تصادفی تطبیقی مقایسه می‌شود. در این دسته از الگوریتم‌ها، هر آزمایه به‌صورت یک یا چند نقطه درون یک یا چند فضای چندبعدی نگاشت می‌شود و میزان تفاوت بین دو آزمایه در واقع از فاصله بین نقاط متناظر آن دو آزمایه محاسبه می‌شود. در نتیجه، در روال کلی الگوریتم‌های انتخاب آزمایه بر اساس محاسبه فاصله، باید نقاط نگاشت‌شده آزمایه‌های انتخاب‌شده را در حین اجرا ذخیره کنیم و به عبارتی، میزان فضای موردنیاز الگوریتم‌های انتخاب آزمایه بر اساس محاسبه فاصله وابسته به تعداد آزمایه‌های خروجی الگوریتم خواهد بود.

هزینه زمانی روش‌های انتخاب آزمایه مبتنی بر محاسبه فاصله، به تعداد آزمایه‌های قبلی انتخاب‌شده بستگی دارد و باید فاصله بین هر آزمایه کاندید و هر آزمایه درون مجموعه آزمایه‌های انتخاب‌شده در مراحل قبلی الگوریتم محاسبه شود. البته به مرور زمان روش‌هایی برای کاهش این فضای مورد استفاده و کوچک‌تر کردن مجموعه آزمایه‌های انتخاب‌شده مراحل قبلی الگوریتم برای ارزیابی آزمایه‌های مجموعه کاندید پیشنهاد شد که در فصل ۲ چند مورد از این روش‌ها ذکر شدند. اما با توجه به روال کلی این استراتژی‌ها، پیاده‌سازی آن‌ها الگوریتم انتخاب آزمایه را ملزم به انجام یک سری عملیات اضافه قبل از انتخاب هر آزمایه از بین آزمایه‌های مجموعه کاندید می‌کند و از طرفی چون باعث کوچک‌تر شدن مجموعه آزمایه‌های انتخاب‌شده برای ارزیابی مجموعه آزمایه‌های کاندید می‌شود، دقت الگوریتم انتخاب آزمایه را نیز کاهش خواهد داد.

\subsection{هزینه فضایی و زمانی استراتژی‌های انتخاب آزمایه مبتنی بر امتیازدهی}

در استراتژی‌های انتخاب آزمایه مبتنی بر امتیازدهی، تنها مؤلفه‌ای که نیاز به ذخیره شدن دارد، مجموعه رفتارهای تاکنون مشاهده‌شده توسط آزمایه‌های انتخاب‌شده در مراحل قبلی است و تعداد این رفتارها محدود و متناهی است.

فرض کنید در یک سیستم تحت آزمون فرضی، $n$ عدد تابع داریم که هر تابع با توجه به تعریف خاص رفتاری که ما استفاده می‌کنیم، به تعداد $w_i$ رفتار مختلف می‌تواند از خود نشان دهد که در قسمت تعریف رفتار بر متناهی و ثابت بودن تعداد این رفتارها برای هر تابع تأکید شده است. در نتیجه، در مجموع تعداد رفتارهای سیستم تحت آزمون $\sum_{i=1}^{n} w_i$ خواهد بود که یک عدد ثابت است. به بیان دیگر، هزینه فضایی الگوریتم‌های انتخاب آزمایه با استفاده از محاسبه امتیاز برخلاف الگوریتم‌های انتخاب آزمایه بر اساس محاسبه فاصله به تعداد آزمایه‌های خروجی الگوریتم وابسته نیست و همواره برای تولید تعداد دلخواهی آزمایه با استفاده از این الگوریتم‌ها، فضای موردنیاز از یک عدد ثابت $W$ بیشتر نخواهد شد.

هزینه زمانی الگوریتم‌های انتخاب آزمایه بر اساس محاسبه امتیاز نیز وابسته به تعداد آزمایه‌های تولید شده نیست. در هر مرحله، آزمایه‌های درون مجموعه کاندید اجرا خواهند شد و رفتارهای مشاهده‌شده توسط آن‌ها در محلی ذخیره خواهد شد و سپس از مقایسه رفتارهای مشاهده‌شده توسط هر آزمایه کاندید و مجموعه رفتارهایی که تاکنون توسط آزمایه‌های انتخاب‌شده مشاهده شده است، امتیاز هر آزمایه محاسبه خواهد شد.

\subsection{جامعیت استراتژی انتخاب آزمایه مبتنی بر امتیازدهی}

استراتژی آزمون تصادفی تطبیقی مبتنی بر امتیازدهی، قابلیت پیاده‌سازی روی هر سیستم تحت آزمونی را دارد و فقط محدود به یک دسته خاص از سیستم‌های نرم‌افزاری نیست. به عبارت دیگر، در این پژوهش یک استراتژی جامع برای پیاده‌سازی روش آزمون تصادفی تطبیقی ارائه شده است که روی انواع مختلف سیستم‌های نرم‌افزاری با ورودی‌ها و خروجی‌های متنوع قابل پیاده‌سازی و استفاده است.




