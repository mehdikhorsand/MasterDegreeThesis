% !TeX root=../main.tex
\chapter{نتایج}

\section{ارزیابی}
برای ارزیابی روش‌های آزمون تصادفی تطبیقی مبتنی بر استراتژی امتیازدهی که در این پژوهش ارائه شده است، روش‌های \lr{ART\_AutoISP} و \lr{ART\_AutoISP\_C} به عنوان نمایندگان این دسته از روش‌ها، با روش‌های \lr{ART\_WTClustering} و \lr{ART\_TFCClustering} که در آخرین پژوهش‌های انجام‌شده در حوزه روش‌های آزمون تصادفی تطبیقی معرفی شده‌اند و طبق مستندات این مقاله از همه روش‌های ارائه‌شده تاکنون عملکرد نسبتاً بهتری دارند و همچنین روش آزمون تصادفی سنتی، مقایسه خواهند شد. نتایج این مقایسه بر اساس معیارهای ارزیابی که در ادامه توضیح داده شده‌اند، تحلیل و بررسی خواهند شد.

\section{معیارهای ارزیابی}

\subsection{معیارهای میزان پوشش}

\subsubsection{پوشش شاخه}

پوشش شاخه \LTRfootnote{Branch Coverage} یکی از معیارهای کلیدی در ارزیابی جامعیت آزمایش‌های نرم‌افزاری است که بر اساس بررسی مسیرهای مختلف اجرای کد تعیین می‌شود. این معیار با هدف اطمینان از آزمایش همه مسیرهای ممکن که از تصمیمات منطقی کد نشأت می‌گیرند، استفاده می‌شود. به عبارت دیگر، پوشش شاخه میزان بررسی و آزمایش تمامی انشعابات (شرایط) موجود در کد را اندازه‌گیری می‌کند.
\begin{itemize}
	
	\item \textbf{فرآیند ارزیابی پوشش شاخه}

فرآیند ارزیابی پوشش شاخه شامل مراحل زیر است:

\begin{enumerate}
	\item \textbf{شناسایی شاخه‌ها:} در این مرحله، تمامی نقاط تصمیم‌گیری در کد (مانند دستورات شرطی \texttt{if-else}، \texttt{switch-case} و حلقه‌ها) شناسایی می‌شوند. هر یک از این نقاط می‌تواند به دو یا چند شاخه ممکن منتهی شود که باید توسط آزمایه‌ها پوشش داده شوند.
	\item \textbf{اجرای آزمایه‌ها:} مجموعه‌ای از آزمایه‌ها طراحی و اجرا می‌شوند تا تمام شاخه‌های شناسایی‌شده را پوشش دهند. هدف این است که برای هر تصمیم‌گیری در کد، تمام شاخه‌های ممکن (مسیرهای \textit{true} و \textit{false}) بررسی شوند.
	\item \textbf{محاسبه پوشش شاخه:} پس از اجرای آزمایه‌ها، میزان پوشش شاخه با محاسبه نسبت شاخه‌های پوشش‌داده‌شده به کل شاخه‌های موجود در کد به دست می‌آید. این معیار معمولاً به صورت درصدی بیان می‌شود:
	\[
	\text{پوشش شاخه (\%)} = \frac{\text{تعداد شاخه‌های اجرا شده}}{\text{تعداد کل شاخه‌ها}} \times 100
	\]
\end{enumerate}

\item \textbf{اهمیت پوشش شاخه}

پوشش شاخه یک معیار قوی‌تر و دقیق‌تر از پوشش خط (\textit{Line Coverage}) محسوب می‌شود، زیرا علاوه بر پوشش دادن تمام خطوط کد، اطمینان حاصل می‌کند که تمامی شاخه‌های ممکن از تصمیمات منطقی کد نیز مورد آزمایش قرار گرفته‌اند. این معیار به ویژه در شناسایی نقاطی از کد که ممکن است در صورت آزمایش نشدن به بروز خطاهای پنهان منجر شوند، اهمیت دارد. پوشش شاخه به توسعه‌دهندگان کمک می‌کند تا با اطمینان بیشتری درستی و کامل بودن آزمایه‌های خود را ارزیابی کنند و نقاط ضعف احتمالی را شناسایی و برطرف کنند.
\end{itemize}

\subsubsection{پوشش جهش}
%\subsection{معیار ارزیابی میزان پوشش جهش (Mutation Coverage)}

یکی از معیارهای مهم در ارزیابی کیفیت و اثربخشی آزمون‌های نرم‌افزاری، معیار پوشش جهش\LTRfootnote{Mutation Coverage} است. این معیار به بررسی توانایی آزمایه‌ها در شناسایی تغییرات عمدی و کوچک (جهش‌ها) در کد منبع نرم‌افزار می‌پردازد. ایده اصلی پشت این معیار، این است که با ایجاد تغییرات جزئی در کد منبع، بتوان قابلیت کشف خطاها توسط مجموعه آزمایه‌ها را ارزیابی کرد.

\begin{itemize}
	\item \textbf{فرآیند ارزیابی پوشش جهش}

فرآیند ارزیابی پوشش جهش شامل مراحل زیر است:

\begin{enumerate}
	\item \textbf{ایجاد جهش‌ها:} در این مرحله، تغییرات کوچکی در کد منبع نرم‌افزار ایجاد می‌شود. این تغییرات به صورت عمدی و هدفمند اعمال می‌شوند تا به اصطلاح «کد جهش‌یافته» تولید شود. برای مثال، ممکن است یک عملگر منطقی مانند \texttt{==} به \texttt{!=} تغییر داده شود.
	\item \textbf{اجرای آزمایه‌ها:} مجموعه آزمایه‌های موجود روی کد جهش‌یافته اجرا می‌شوند. هدف این است که آزمایه‌ها بتوانند این جهش‌های عمدی را شناسایی کرده و با شکست مواجه شوند، یعنی خروجی آزمایه‌ها پس از اعمال جهش با خروجی اصلی متفاوت باشد.
	\item \textbf{محاسبه پوشش جهش:} اگر آزمایه‌ها بتوانند جهش‌ها را شناسایی کنند و خروجی‌های نادرست ایجاد شده توسط کد جهش‌یافته را تشخیص دهند، گفته می‌شود که آزمایه موفق به «کشتن» آن جهش شده است. میزان پوشش جهش به صورت درصدی از جهش‌های کشته‌شده به کل جهش‌های اعمال‌شده محاسبه می‌شود:
	\[
	\text{پوشش جهش (\%)} = \frac{\text{تعداد جهش‌های کشته‌شده توسط آزمایه‌ها}}{\text{تعداد کل جهش‌ها}} \times 100
	\]
\end{enumerate}

	\item \textbf{اهمیت پوشش جهش}

پوشش جهش به عنوان یک معیار مکمل برای سایر معیارهای پوشش کد، مانند پوشش خط\LTRfootnote{Line Coverage} یا پوشش شاخه\LTRfootnote{Branch Coverage}، مورد استفاده قرار می‌گیرد. این معیار می‌تواند به شناسایی ضعف‌های موجود در مجموعه آزمایه‌ها کمک کند و نقاطی از کد را که به درستی آزمون نشده‌اند، نمایان سازد. در نهایت، پوشش جهش کمک می‌کند تا مطمئن شویم که آزمایه‌های طراحی‌شده به اندازه کافی جامع هستند و می‌توانند تغییرات نامطلوب و خطاهای پنهان را در کد نرم‌افزار کشف کنند.

\end{itemize}

\subsection{معیارهای میزان قدرت تشخیص خطا}
معیارهای ارزیابی \lr{F-measure} و \lr{F-time} به‌طور گسترده برای تحلیل قدرت تشخیص خطای روش‌های آزمون تصادفی تطبیقی در پژوهش‌های متعددی که تاکنون در این زمینه ارائه شده‌اند، مورد استفاده قرار گرفته‌اند. هر چه مقدار این معیارها در یک روش آزمون تصادفی تطبیقی کمتر باشد، نشان‌دهنده قابلیت بالاتر آن روش در تشخیص خطاها است.


\subsubsection{معیار ارزیابی \lr{F-measure}}

در آزمون تصادفی تطبیقی، معیار \lr{F-Measure} به مدت زمانی اشاره دارد که طول می‌کشد تا اولین خطا در سیستم تحت آزمون کشف شود. این معیار به‌طور خاص بر ارزیابی کارایی روش آزمون تمرکز دارد، به این صورت که مشخص می‌کند روش به‌کاررفته چه تعداد آزمایه نیاز دارد که تولید کند تا بتواند اولین خطا را در سیستم تحت آزمون شناسایی کند. F-Measure به عنوان یک شاخص کلیدی در ارزیابی روش‌های آزمون تصادفی تطبیقی به کار می‌رود. روش‌های آزمونی که دارای مقدار \lr{F-Measure} کوچکتری هستند، به عنوان روش‌های کارآمدتر شناخته می‌شوند زیرا قادر به کشف خطاها در تعداد آزمایه کمتری بوده‌اند.

\subsubsection{معیار ارزیابی F-Time}

\lr{F-Time} نیز یکی دیگر از معیارهای مهم در ارزیابی روش‌های آزمون تصادفی تطبیقی است. این معیار به زمان لازم برای کشف اولین خطا در سیستم تحت آزمون اشاره دارد. \lr{F-Time} به‌طور دقیق‌تر نشان می‌دهد که روش آزمون چه مدت زمانی نیاز دارد تا اولین خطا را در شرایط واقعی کشف کند.

این معیار به‌عنوان معیاری برای اندازه‌گیری سرعت عملکرد روش‌های آزمون تصادفی تطبیقی به کار می‌رود. روش‌هایی که زمان کمتری برای کشف اولین خطا نیاز دارند، معمولاً کارآمدتر محسوب می‌شوند.


%\thispagestyle{empty} 
%\label{chap:results}
%\section{مقدمه} 
%ارائهٔ داده‌ها، نتایج، تحلیل و تفسیر اولیهٔ آنها در این فصل ارائه می‌شود. در ارائهٔ نتایج با توجه به راهنمای کلی نگارش فصل‌ها، تا حد امکان، ترکیبی از نمودار و جدول استفاده شود. با توجه به حجم و ماهیت تحقیق و با صلاحدید استاد راهنما، این فصل می‌تواند تحت عنوانی دیگر بیاید. در صورتی که حجم داده‌ها زیاد باشد، بهتر است به صورت نمودار یا در قالب ضمیمه ارائه نشده و فقط نمونه‌ها در متن آورده شود. در این فصل باید به سوالات تحقیق، عطف به یافته‌های محقق، پاسخ داده شود. اگر تحقیق دارای آزمون فرض باشد، پذیرش یا عدم پذیرش فرضیه‌ها در این فصل گزارش می‌شود. این فصل حدود ۴۰ صفحه است.
%
%\section{محتوا}
%در این بخش به سوالات تحقیق، بر اساس داده‌ها و یافته‌های محقق، پاسخ داده می‌شود. داده‌ها با فرمت مناسبی ارائه می‌شوند؛ مدل (ها) اجرا شده و نتیجه آن مشخص می‌شود.
%
%\section{اعتبارسنجی}
%از طریق مقایسهٔ نتایج با نتایج کارهای دیگران، استفاده از روش‌های تحلیل پایائی
%\lr{(reliability)}
%و اعتبار
%\lr{(validity)}،
%نظرگیری از خبرگان
%\lr{(expert judgment or feedback)}
%و یا
%\lr{triangulation}
%انجام می‌شود.
